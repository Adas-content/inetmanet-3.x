\chapter{Point-to-Point Links}
\label{cha:ppp}


\section{Overview}

The INET Framework contains an implementation of the Point-to-Point Protocol
as described in RFC1661 with the following limitations:

\begin{itemize}
\item There are no LCP messages for link configuration,
link termination and link maintenance.
The link can be configured by setting module parameters.
\item PFC and ACFC are not supported, the PPP frame
always contains the 1-byte Address and Control fields
and a 2-byte Protocol field.
\item PPP authentication is not supported
\item Link quality monitoring protocols are not supported.
\item There are no NCP messages, the network layer protocols are
configured by other means.
\end{itemize}

The modules of the PPP model can be found in the \nedtype{inet.linklayer.ppp}
package:

\begin{description}
\item[\nedtype{PPP}] This simple module performs encapsulation
of network datagrams into PPP frames and decapsulation of
the incoming PPP frames. It can be connected to the network
layer directly or can be configured to get the outgoing messages
from an output queue. The module collects statistics about
the transmitted and dropped packages.

\item[\nedtype{PPPInterface}] is a compound module complementing
the \nedtype{PPP} module with an output queue. It implements
the \nedtype{IWiredNic} interface. Input and output hooks can be configured
for further processing of the network messages.

\end{description}

\section{PPP frames}

According to RFC1662 the PPP frames contain the following fields:

\begin{bytefield}[bitheight=2\baselineskip]{40}
\bitbox{8}{Flag \\ 01111110} &
\bitbox{8}{Address \\ 11111111} &
\bitbox{8}{Control \\ 00000011} \\
\bitbox{8}{Protocol \\ 8/16 bits} &
\bitbox{10}{Information \\ \textasteriskcentered } &
\bitbox{8}{Padding \\ \textasteriskcentered } \\
\bitbox{8}{FCS \\ 16/32 bits}
\bitbox{8}{Flag \\ 01111110 } &
\bitbox[ltb]{14}{Inter-frame Fill \\ or next Address}
\end{bytefield}

The corresponding message type in the INET framework is \msgtype{PPPFrame}.
It contains the Information field as an encapsulated \cppclass{cMessage}
object. The Flag, Address and Control fields are omitted
from \msgtype{PPPFrame} because they are constants. The FCS field is
omitted because no CRC computed during the simulation, the bit error
attribute of the \cppclass{cMessage} used instead. The Protocol
field is omitted because the protocol is determined from the class
of the encapsulated message.

The length of the PPP frame is equal to the length of the encapsulated
datagram plus 7 bytes. This computation assumes that
\begin{itemize}
\item there is no inter-octet time fill, so only one Flag sequence
needed per frame
\item padding is not applied
\item PFC and ACFC compression is not applied
\item FCS is 16 bit
\item no escaping was applied
\end{itemize}

\section{PPP module}

The PPP module receives packets from the upper layer in the \fvar{netwIn}
gate, encapsulates them into \msgtype{PPPFrame}s, and send it to the
physical layer through the \fvar{phys} gate. The \msgtype{PPPFrame}s
received from the \fvar{phys} gate are decapsulated and sent to the upper
layer immediately through the \fvar{netwOut} gate.

Incoming datagrams are waiting in a queue if the line is currently busy.
In routers, PPP relies on an external queue module (implementing
\nedtype{IOutputQueue}) to model finite buffer, implement QoS and/or RED,
and requests packets from this external queue one-by-one. The name
of this queue is given as the \fpar{queueModule} parameter.

In hosts, no such queue is used, so \nedtype{PPP} contains an internal
queue named txQueue to queue up packets wainting for transmission.
Conceptually txQueue is of inifinite size, but for better diagnostics
one can specify a hard limit in the \fpar{txQueueLimit} parameter -- if
this is exceeded, the simulation stops with an error.

The module can be used in simulations where the nodes are connected and
disconnected dinamically. If the channel between the PPP modules is down,
the messages received from the upper layer are dropped (including the messages
waiting in the queue). When the connection is restored it will
poll the queue and transmits the messages again.

The PPP module registers itself in the interface table of the node.
The \fvar{mtu} of the entry can be specified by the
\fpar{mtu} module parameter. The module checks the state of the physical link
and updates the entry in the interface table.
% FIXME: The module does not notice if the datarate of the channel changed!
%        It should update the interface entry.

The node containing the PPP module must also contain a
\nedtype{NofiticationBoard} component. Notifications are sent when
transmission of a new PPP frame started (\verb!NF_PP_TX_BEGIN!), finished
(\verb!NF_PP_TX_END!) or when a PPP frame received (\verb!NF_PP_RX_END!).

The PPP component is the source of the following signals:
\begin{itemize}
\item \tbf{txState} state of the link (0=idle,1=busy)
\item \tbf{txPkBytes} number of bytes transmitted
\item \tbf{rxPkBytesOk} number of bytes received successfully
\item \tbf{droppedPkBytesBitError} number of bytes received in erronous frames
\item \tbf{droppedPkBytesIfaceDown} number of bytes dropped because the link is down
\item \tbf{rcvdPkBytesFromHL} number of bytes received from the the upper layer
\item \tbf{passedUpPkBytes} number of bytes sent to the the upper layer
\end{itemize}

These signals are recorded as statistics (sum, count and vector), so
they can be analyzed after the simulation.

When the simulation is executed with the graphical user interface
the module displays useful statistics. If the link is operating,
the datarate and number of received, sent and dropped messages
show in the tooltip. When the link is broken, the number of dropped
messages is displayed. The state of the module is indicated by the color
of the module icon and the connection (yellow=transmitting).

\section{PPPInterface module}

The \nedtype{PPPInterface} is a compound module implementing the \nedtype{IWiredNic}
interface. It contains a \nedtype{PPP} module and a passive queue for the messages
received from the network layer.

The queue type is specified by the \fpar{queueType} parameter.
It can be set to \nedtype{NoQueue} or to a module type implementing
the \nedtype{IOutputQueue} interface. There are implementations
with QoS and RED support.

In typical use of the \nedtype{PPP} module it is augmented with other nodes
that monitor the traffic or simulate package loss and duplication.
The \nedtype{PPPInterface} module abstract that usage by adding
\nedtype{IHook} components to the network input and output of the
\nedtype{PPP} component. Any number of hook can be added by
specifying the \fpar{numOutputHooks} and \fpar{numInputHooks}
parameters and the types of the \fvar{outputHook} and \fvar{inputHook}
components. The hooks are chained in their numeric order.


%%% Local Variables:
%%% mode: latex
%%% TeX-master: "usman"
%%% End:


