\chapter{The MPLS Models}
\label{cha:mpls}


\section{Overview}

Blah blah blah


\section{MPLS/RSVP/LDP Model - Implemented Standards}

The implementation follows those RFCs below:

\begin{itemize}
  \item RFC 2702: Requirements for Traffic Engineering Over MPLS
  \item RFC 2205: Resource ReSerVation Protocol
  \item RFC 3031: Multiprotocol Label Switching Architecture
  \item RFC 3036: LDP Specification
  \item RFC 3209: RSVP-TE Extension to RSVP for LSP tunnels
  \item RFC 2205: RSVP Version 1 - Functional Specification
  \item RFC 2209: RSVP Message processing Version 1
\end{itemize}

\section{MPLS Operation}

The following algorithm is carried out by the MPLS module:

\begin{verbatim}
Step 1: - Check which layer the packet is coming from
Alternative 1: From layer 3
    Step 1: Find and check the next hop of this packet
    Alternative 1: Next hop belongs to this MPLS cloud
        Step 1: Encapsulate the packet in an MPLS packet with
        IP_NATIVE_LABEL label
        Step 2: Send to the next hop
        Step 3: Return
    Alternative 2: Next hop does not belong to this MPLS cloud
        Step 1: Send the packet to the next hop
Alternative 2: From layer 2
    Step 1: Record the packet incoming interface
    Step 2: - Check if the packet is for this LSR
    Alternative 1: Yes
        Step 1: Check if the packet has label
        Alternative 1: Yes
            Step 1: Strip off all labels and send the packet to L3
            Step 2: Return
        Alternative 2: No
            Step 1: Send the packet to L3
            Step 2: Return
    Alternative 2: No
        Step 1: Continue to the next step
    Step 3: Check the packet type
    Alternative 1: The packet is native IP
        Step 1: Check the LSR type
        Alternative 1: The LSR is an Ingress Router
            Step 1: Look up LIB for outgoing label
            Alternative 1: Label cannot be found
                Step 1: Check if the label for this FEC is being requested
                Alternative 1: Yes
                    Step 1: Return
                Alternative 2: No
                    Step 1: Store the packet with FEC id
                    Step 2: Do request the signalling component
                    Step 3: Return
            Alternative 2: Label found
                Step 1: Carry out the label operation on the packet
                Step 2: Forward the packet to the outgoing interface found
                Step 3: Return
        Alternative 2: The LSR is not an Ingress Router
            Step 1: Print out the error
            Step 2: Delete the packet and return
    Alternative 2: The packet is MPLS
        Step 1: Check the LSR type
        Alternative 1: The LSR is an Egress Router
            Step 1: POP the top label
            Step 2: Forward the packet to the outgoing interface found
            Step 3: Return
        Alternative 2: The LSR is not an Egress Router
            Step 1: Look up LIB for outgoing label
            Alternative 1: Label cannot be found
                Step 1: Check if the label for this FEC is being requested
                Alternative 1: Yes
                    Step 1: Return
                Alternative 2: No
                    Step 1: Store the packet with FEC id
                    Step 2: Do request the signalling component
                    Step 3: Return
            Alternative 2: Label found
                Step 1: Carry out the label operation on the packet
                Step 2: Forward the packet to the outgoing interface found
                Step 3: Return
Step 2: Return
\end{verbatim}


\section{LDP Message Processing}

The simulation follows message processing rules specified in RFC3036
(LDP Specification). A summary of the algorithm used in the RFC is
presented below.

\subsection{Label Request Message processing}

An LSR may transmit a Request message under any of the conditions below:

\begin{itemize}
  \item The LSR recognizes a new FEC via the forwarding tale, and the next hop
    is its LDP peer. The LIB of this LSR does not have a mapping from the
    next hop for the given FEC.
  \item Network topology changes, the next hop to the FEC is no longer valid
    and new mapping is not available.
  \item The LSR receives a Label Request for a FEC from an upstream LDP and it
    does not have label binding information for this FEC. The FEC next hop
    is an LDP peer.
\end{itemize}

Upon receiving a Label Request message, the following procedures will be
performed:

\begin{verbatim}
Step 1: Extract the FEC from the message and locate the incoming interface
        of the message.
Step 2: Check whether the FEC is an outstanding FEC.
    Alternative 1: This FEC is outstanding
        Step 1: Return
    Alternative 2: This FEC is not outstanding
        Step 1: Continue
Step 3: Check if there is an exact match of the FEC in the routing table.
    Alternative 1: There is an exact match
        Step 1: Continue
    Alternative 2: There is no match
        Step 1: Construct a Notification message of No route and
                send this message back to the sender.
Step 4: Make query to local LIB to find out the corresponding label.
    Alternative 1: The label found
        Step 1: Construct a Label Mapping message and send over
                the incoming interface.
    Alternative 2: The label cannot be found for this FEC
        Step 1: Construct a new Label Request message and send
                the message out using L3 routing.
        Step 2: Construct a Notification message indicating that the
                label cannot be found.
\end{verbatim}

\subsection{Label Mapping Message processing}

Upon receiving a Label Mapping message, the following procedures will be
performed:

\begin{verbatim}
Step 1: Extract the FEC and the label from the message.
Step 2: Check whether this is an outstanding FEC
    Alternative 1: This FEC is outstanding
        Step 1: Continue
    Alternative 2: This FEC is not outstanding
        Step 1: Send back the server an Notification of Error message.
Step 3: Install the new label to the local LIB using the extracted label,
        FEC and the message incoming interface.
\end{verbatim}


\section{LIB Table File Format}

The format of a LIB table file is:

The beginning of the file should begin with comments. Lines that begin with \# are treated
as comments. An empty line is required after the comments. The "LIB TABLE"
syntax must come next with an empty line. The column headers follow. This header
must be strictly "In-lbl In-intf Out-lbl Out-intf". Column
values are after that with space or tab for field separation.
The following is a sample of lib table file.

\begin{verbatim}
#lib table for MPLS network simulation test
#lib1.table for LSR1 - this is an edge router
#no incoming label for traffic from in-intf 0 &1 - LSR1 is ingress router for those traffic
#no outgoing label for traffic from in_intf 2 &3 - LSR 1 is egress router for those traffic

LIB TABLE:

In-lbl  In-intf         Out-lbl     Out-intf
1       193.233.7.90    1           193.231.7.21
2       193.243.2.1     0           193.243.2.3
\end{verbatim}

\section{The CSPF Algorithm}

CSPF stands for Constraint Shortest Path First.
This constraint-based routing is executed online by Ingress Router.
The CSPF calculates an optimum explicit route (ER), based on
specific constraints. CSPF relies on a Traffic Engineering Database (TED)
to do those calculations. The resulting route is then used by RSVP-TE.

The CSPF in particular and any constraint based routing process requires following
inputs:

\begin{itemize}
  \item Attributes of the traffic trunks, e.g., bandwidth, link affinities
  \item Attributes of the links of the network, e.g. bandwidth, delay
  \item Attributes of the LSRs, e.g. types of signaling protocols supported
  \item Other topology state information.
\end{itemize}

There has been no standard for CSPF so far. The implementation of CSPF in
the simulation is based on the concept of "induced graph" introduced in RFC
2702. An induced graph is analogous to a virtual topology in an overlay
model. It is logically mapped onto the physical network through the
selection of LSPs for traffic trunks. CSPF is similar to a normal SPF,
except during link examination, it rejects links without capacity or links
that do not match color constraints or configured policy. The CSPF
algorithm used in the simulation has two phases. In the first phase, all
the links that do not satisfy the constraints of bandwidth are excluded
from the network topology. The link affinity is also examined in this
phase. In the second phase, Dijkstra algorithm is performed.

Dijkstra Algorithm:

\begin{verbatim}
Dijkstra(G, w, s):
   Initialize-single-source(G,s);
   S = empty set;
   Q = V[G];
   While Q is not empty {
       u = Extract-Min(Q);
       S = S union {u};
       for each vertex v in Adj[u] {
           relax(u, v, w);
       }
   }
\end{verbatim}

In which:
\begin{itemize}
  \item G: the graph, represented in some way (e.g. adjacency list)
  \item w: the distance (weight) for each edge (u,v)
  \item s (small s): the starting vertex (source)
  \item S (big S): a set of vertices whose final shortest path from s have already been determined
  \item Q: set of remaining vertices, Q union S = V
\end{itemize}

\section{The traffic.xml file}

The traffic.xml file is read by the RSVP-TE module (RSVP).
The file must be in the same folder as the executable
network simulation file.

The XML elements used in the "traffic.xml" file:

\begin{itemize}
  \item \ttt{<Traffic></Traffic>} is the root element. It may contain one or more \ttt{<Conn>} elements.
  \item \ttt{<Conn></Conn>} specifies an RSVP session. It may contain the following elements:
  \begin{itemize}
    \item \ttt{<src></src>} specifies sender IP address
    \item \ttt{<dest></dest>} specifies receiver IP address
    \item \ttt{<setupPri></setupPri>} specifies LSP setup priority
    \item \ttt{<holdingPri></holdingPri>} specifies LSP holding priority
    \item \ttt{<bandwidth></bandwidth>} specifies the requested BW.
    \item \ttt{<delay></delay>} specifies the requested delay.
    \item \ttt{<route></route>} specifies the explicit route. This is a comma-separated
      list of IP-address, hop-type pairs (also separated by comma).
      A hop type has a value of 1 if the hop is a loose hop and 0 otherwise.
  \end{itemize}
\end{itemize}

The following presents an example file:

\begin{verbatim}
<?xml version="1.0"?>
<!-- Example of traffic control file -->
<traffic>
   <conn>
       <src>10.0.0.1</src>
       <dest>10.0.1.2</dest>
       <setupPri>7</setupPri>
       <holdingPri>7</holdingPri>
       <bandwidth>400</bandwidth>
       <delay>5</delay>
   </conn>
   <conn>
       <src>11.0.0.1</src>
       <dest>11.0.1.2</dest>
       <setupPri>7</setupPri>
       <holdingPri>7</holdingPri>
       <bandwidth>100</bandwidth>
       <delay>5</delay>
   </conn>
</traffic>
\end{verbatim}

An example of using RSVP-TE as signaling protocol can be found in
ExplicitRouting folder distributed with the simulation. In this
example, a network similar to the network in LDP-MPLS example is
setup. Instead of using LDP, "signaling" parameter is set to 2 (value
of RSVP-TE handler). The following xml file is used for traffic
control. Note the explicit routes specified in the second connection.
It indicates that the route is a strict one since the values of every
hop types are 0. The route defined is 10.0.0.1 -> 1.0.0.1 ->
10.0.0.3 -> 1.0.0.4 -> 10.0.0.5 -> 10.0.1.2.

\begin{verbatim}
<?xml version="1.0"?>
<!-- Example of traffic control file -->
<traffic>
    <conn>
        <src>10.0.0.1</src>
        <dest>10.0.1.2</dest>
        <setupPri>7</setupPri>
        <holdingPri>7</holdingPri>
        <bandwidth>0</bandwidth>
        <delay>0</delay>
        <ER>false</ER>
    </conn>
    <conn>
        <src>11.0.0.1</src>
        <dest>11.0.1.2</dest>
        <setupPri>7</setupPri>
        <holdingPri>7</holdingPri>
        <bandwidth>0</bandwidth>
        <delay>0</delay>
        <ER>true</ER>
        <route>1.0.0.1,0,1.0.0.3,0,1.0.0.4,0,1.0.0.5,0,10.0.1.2,0</route>
    </conn>
</traffic>
\end{verbatim}

%%% Local Variables:
%%% mode: latex
%%% TeX-master: "usman"
%%% End:


