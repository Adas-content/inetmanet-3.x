\chapter{History}
\label{cha:History}

\section{IPSuite to INET Framework (2000-2006)}

The predecessor of the INET framework was written by Klaus
Wehrle, Jochen Reber, Dirk Holzhausen, Volker Boehm, Verena Kahmann,
Ulrich Kaage and others at the University of Karlsruhe during 2000-2001,
under the name IPSuite.

The MPLS, LDP and RSVP-TE models were built as an add-on to IPSuite
during 2003 by Xuan Thang Nguyen (Xuan.T.Nguyen@uts.edu.au) and other
students at the University of Technology, Sydney under supervision of
Dr Robin Brown. The package consisted of around 10,000 LOCs, and was
published at http://charlie.it.uts.edu.au/~tkaphan/xtn/capstone (now
unavailable).

After a period of IPSuite being unmaintained, Andras Varga took over
the development in July 2003. Through a series of snapshot releases in
2003-2004, modules got completely reorganized, documented, and many of them
rewritten from scratch. The MPLS models (including RSVP-TE, LDP, etc)
also got refactored and merged into the codebase.

During 2004, Andras added a new, modular and extensible TCP implementation,
application models, Ethernet implementation and an all-in-one IP model
to replace the earlier, modularized one.

The package was renamed INET Framework in October 2004.

Support for wireless and mobile networks got added during summer 2005
by using code from the Mobility Framework.

The MPLS models (including LDP and RSVP-TE) got revised and mostly
rewritten from scratch by Vojta Janota in the first half of 2005
for his diploma thesis. After further refinements by Vojta, the new code
got merged into the INET CVS in fall 2005, and got eventually released
in the March 2006 INET snapshot.

The OSPFv2 model was created by Andras Babos during 2004 for his diploma
thesis which was submitted early 2005. This work was sponsored by Andras Varga,
using revenues from commercial OMNEST licenses. After several refinements and fixes,
the code got merged into the INET Framework in 2005, and became part of the
March 2006 INET snapshot.

The Quagga routing daemon was ported into the INET Framework also by Vojta
Janota. This work was also sponsored by Andras Varga. During fall 2005 and
the months after, ripd and ospfd were ported, and the methodology of porting
was refined. Further Quagga daemons still remain to be ported.

Based on experience from the IPv6Suite (from Ahmet Sekercioglu's group at
CTIE, Monash University, Melbourne) and IPv6SuiteWithINET (Andras's effort
to refactor IPv6Suite and merge it with INET early 2005), Wei Yang Ng
(Monash Uni) implemented a new IPv6 model from scratch for the
INET Framework in 2005 for his diploma thesis, under guidance from Andras
who was visiting Monash between February and June 2005. This IPv6 model
got first included in the July 2005 INET snapshot, and gradually refined
afterwards.

The SCTP implementation was contributed by Michael Tuexen, Irene Ruengeler
and Thomas Dreibholz

Support for Sam Jensen's Network Simulation Cradle,
which makes real-world TCP stacks available in simulations was added
by Zoltan Bojthe in 2010.

TCP SACK and New Reno implementation was contributed by Thomas Reschka.

Several other people have contributed to the INET Framework by providing
feedback, reporting bugs, suggesting features and contributing patches;
I'd like to acknowledge their help here as well.



%%% Local Variables:
%%% mode: latex
%%% TeX-master: "usman"
%%% End:

